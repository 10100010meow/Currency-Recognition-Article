\section{Conclusions}\label{sec:conclusions}

The proposed recognition system achieved very good detection of banknotes even when they have significant perspective distortion or are partially occluded. It also was robust enough to handle folded and wrinkled banknotes with different kinds of illumination.

This was achieved by careful identifying the regions of the banknotes that had unique features, in order to avoid the usage of structures that are similar between banknotes. This technique in conjunction with the use of reference images with several levels of detail, were crucial to improve the correct matching of keypoints descriptors and ensure the correct recognition of the banknotes.

The system was configured to recognize Euro banknotes, but can easily be reconfigured to detect other currencies.
The achieved results make it a viable option to be used by visually impaired people or to improve automatic banknote counting machines and even increase the security of \glspl{atm} by detecting counterfeit banknotes.
