\section{Conclusions}\label{sec:conclusions}

The proposed recognition system successfully recognized the 95 banknotes in the 80 test images, even when they had significant perspective distortion or were partially occluded. It was also robust enough to handle folded and wrinkled banknotes with different kinds of illumination. This was achieved by carefully identifying the regions of the banknotes that had unique features in order to avoid the usage of structures that were similar between banknotes. This technique in conjunction with the usage of reference images with several levels of detail were crucial to improve the correct matching of keypoints descriptors and ensure the correct recognition of the banknotes. The system was configured to recognize Euro banknotes, but can easily be reconfigured to detect other currencies. The achieved results make it a viable option to be used by visually impaired people or to improve automatic banknote counting machines and even increase the security of \glspl{atm} by detecting counterfeit banknotes.

Future work would include the test of the system using the banknotes under ultra-violet and infra-red light in order to detect with higher confidence counterfeit banknotes and also integrate the system with a speech synthesizer in order to be usable by blind people.
