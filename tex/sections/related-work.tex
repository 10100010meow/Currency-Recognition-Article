\section{Related Work}\label{sec:related-work}

There are several approaches that can be used to successfully recognize banknotes \cite{Pawade2013}, and they range from simple but less robust techniques to more advanced and accurate systems.

The simplest technique is to use template matching to try to find the banknotes in the image by simple bitmap comparison. But this has the problem that both the reference banknotes and the targets in the image must have the same size and perspective view. To mitigate this restriction, the comparison could be done in several scales and orientations, but it wouldn't be very efficient. The dynamic template matching proposed in \cite{Nishimura2009} could be used instead, but it still isn't the best way to handle the recognition.

Other way to tackle this problem would be to perform color and shape segmentation of specific parts of the banknotes, like the method proposed in \cite{Solymar2011}. But this can complicate the implementation since it would have to be tuned to each specific banknote, and it would require a lot of effort to successfully recognize all banknotes from both sides.

A more broad implementation could use the size, color and texture of each banknote to perform the recognition \cite{Hassanpour2007}, but like the previous case, it would require fine tuning to recognize each banknote, and would have to take in consideration the accuracy of the distance measurements, because several banknotes may have similar sizes.

A more robust implementation could use Principal Component Analysis or even adapt the eigenfaces algorithm to try to recognize the banknotes \cite{Grijalva2010}, but this can have some problems when the perspective of the reference banknotes is very different from the ones in the image.

For detection of counterfeit banknotes, ultra violet or infra-red light could be used to highlight specific parts of the banknotes that are hard to duplicate and easier to recognize \cite{Kim2013}. Other similar technique takes advantage of the fact that specific parts of the banknotes are highlighted when they are illuminated with LEDs with different colors and intensities \cite{Radvanyi2011}. And another approach takes in consideration the electromagnetic fields present in sections of the banknotes to perform the recognition \cite{Qian2011}. But all these techniques require special hardware that is too expensive. Moreover, they are not meant to be used by visually impaired people.

Some of the most common techniques used to perform banknote recognition rely on machine learning algorithms, such as Support Vector Machines \iftoggle{ebib}{\cite{Yeh2011,Chang2007,Sun2008A}}{\cite{Yeh2011}}, Artificial Neural Networks \iftoggle{ebib}{\cite{Lee2004,Gai2013,Sun2008B}}{\cite{Gai2013}}, and Hidden Markov Models \iftoggle{ebib}{\cite{Hassanpour2009,Shan2009}}{\cite{Hassanpour2009}}. These techniques usually apply some sort of clustering of features before training the classifier, such as the Bag of Keypoints model \cite{Csurka2004}, or try to extract relevant features from the reference images. After the training, the classifiers can be used to recognize the banknotes. Although this is a good approach to general recognition, it may not be very precise in calculating the exact location and contour of the banknotes.

After the state of art review, it was considered that the technique that was likely to have the best results, and could be efficiently implemented, had to rely in algorithms that detected important features in the reference banknotes. Moreover, these features should be able to be correctly matched in the test images even if the banknotes were in different perspectives and in a wide range of lighting conditions. As such, an approach based in detection of edges \cite{Shi1994}, corners \cite{Rosten2006}, blobs \cite{Matas2004} and even ridges, would yield the identification of keypoints that could be successfully matched in the conditions presented earlier. For this matching to succeed, a scale and rotation invariant descriptor should be computed for each keypoint, using for example \gls{sift} \cite{Lowe2004} or \gls{surf} \cite{Bay2006} algorithms. After this matching, an outlier removal step could be applied to improve the accuracy of the detection, and the final recognition could be analyzed to make sure it was really recognized a banknote. This kind of approach has proved that it can achieved very good results, as shown in \iftoggle{ebib}{\cite{Hasanuzzaman2011,Hasanuzzaman2012,Toytman2011}}{\cite{Hasanuzzaman2012}}.
