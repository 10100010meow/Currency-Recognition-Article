\section{Analysis of Results}\label{sec:results-analysis}

Several configurations were tested by combining different feature detectors with feature descriptors, and also using different approaches to perform the descriptor matching and inliers filtering. The best results were achieved with \gls{sift} as detector and descriptor in conjunction with a brute force matcher and a global match inliers filtering method.

The best performance of \gls{sift} is mainly related to the fact that it can select feature points that can be reliably detected even if the objects are in different perspective views. Moreover, it can compute descriptors that are robust to different lighting conditions.

For real-time use, the \gls{surf} feature detector and descriptor is more suitable, since it achieves similar results with significant less computation time. This is achieved mainly due to the simplification of the computations by using integral images.

The brute force matcher, although slower than \gls{flann}, achieved better results because unlike the heuristic approach, it matches all descriptors in order to find the best correspondences. However, if the system is to be used in real time, \gls{flann} can be employed with very similar results and lower computation time.

In relation to the inliers filtering method, the best results were achieved when the best matched reference image was selected based on the global inliers ratio. However for test images in which most of the banknotes regions were occluded, the local inliers ratio performed better. This occurred because the local matching of patches avoids the removal of recognition results that have low inliers ratio but in which these inliers are clustered in a small area of the banknote.
