\section{Introduction}\label{sec:introduction}

Banknotes play a critical role in our trading society, and although digital currency is becoming more popular, physical banknotes still account for a large amount of the local transactions. As such, systems that are able to recognize banknotes can be applied to aid in the manipulation of this type of currency.

These kind of systems are critical to visually impaired people, since they allow them to be more independent while avoiding the help of untrusted people. Also, they can increase the security and reliability of ATMs [1], by making sure the maintenance operations are performed correctly and the valid banknotes are not replaced with counterfeits. Other less critical applications are related with automatic sorting and counting of banknotes to speedup transactions and money transfers.

For these types of systems to be effective and useful, they must be able to recognize the banknotes in several perspective views, scale dimensions, and should also tolerate cluttered environments with different lighting conditions. Besides these critical requirements, in order to be properly used to help visually impaired people, they should also be able to recognize folded, wrinkled and worn banknotes.
For the implementation of the robust banknote recognition system, the input images are preprocessed to remove environment noise and improve contrast and brightness. Then important keypoints and their associated descriptors are computed, to later be used to find the best matching in a database of valid banknotes. The correct matching of keypoint descriptors is critical to ensure the proper recognition of the banknotes. As such, methods to filter the inliers from the matches are employed. There are several techniques to perform such filtering, such as the ratio test (presented in section 7.1 of [2]) and the homography outlier removal (chapter 3 of [3]). Although these techniques can yield very good results, a postprocessing analysis is applied to make sure the results obtained are really banknotes. This is related to the fact that the matching of several parts of wrinkle banknotes may result in the recognition of multiple instances of the same banknote. In addition, images similar to banknotes or from other countries banknotes may yield incorrect partial matches. As such, this postprocessing phase is critical to ensure the correct recognition of the banknotes. It starts by computing the banknote contour using the retrieved homography, and then removes any result that has a convex contour, or has its area, circularity and aspect ratio outside the acceptable ranges for banknotes. To detect multiple banknotes in the same image, the inliers of the last recognized banknote are removed and the process presented earlier is repeated until there are no more valid matches.

In the following section it will be presented an overview of the several approaches that can be used to perform banknote recognition. In section III a detailed description of the implementation will be provided. In section IV the representative results of the recognition system will be provided and in section V it will be discussed the robustness of the system.
