\begin{abstract}

Robust banknote recognition in different perspective views and in dynamic lighting conditions is a critical component in assistive systems for visually impaired people. It also has an important role in improving the security of ATM maintenance procedures and in increasing the confidence in the results computed by automatic banknote counting machines. Moreover, with the proper hardware, it can be an effective way to detect counterfeit banknotes. With these applications in mind, it was developed a system that can recognize multiple banknotes in different perspective views and scales, even when they are part of cluttered environments in which the lighting conditions may vary considerably. The system is also able to recognize banknotes that are partially visible, folded, wrinkled or even worn by usage. To accomplish this task, the system is based in image processing algorithms, such as feature detection, description and matching. To improve the confidence in the recognition results, the contour of the banknotes is computed using a homography, and its shape is analyzed to make sure that it belongs to a banknote. The system was tested with 82 test images, and all Euro banknotes were successfully recognized, even when there were several banknotes in the same test image, and they were partially occluded.

\keywords{banknote recognition, feature detection, feature description, feature matching, inliers filtering, multiview recognition, noise reduction, shape analysis}

\end{abstract}
